%!BIB program=biber

\documentclass[twocolumn]{article}
\usepackage{geometry}
\usepackage{fancyhdr}
\usepackage{amsmath}
\usepackage{cases}
\usepackage{graphicx}
\usepackage{subfigure} 
\usepackage{array}
\usepackage{unicode-math}
\usepackage[UTF8]{ctex}
\usepackage[hidelinks]{hyperref}
\usepackage{gbt7714}
\usepackage{doi}
\usepackage{xcolor}
\usepackage{algorithm2e} %伪代码宏
\usepackage{booktabs} %画三线表
\usepackage{arydshln} %添加虚线
\usepackage{makecell}
%\renewcommand{\vec}[1]{\boldsymbol{#1}}

\setmathfont{Cambria Math}
\geometry{a4paper,left=0.5cm,right=0.5cm,top=0.5cm,bottom=0.5cm}

\begin{document}
\footnotesize

\section{基础知识}

\subsection{二进制十进制转换}
例如$(53.7)_{10}=(110101.1\overline{0110})$,小数点左边倒序,小数点右边正序。
\begin{align*}
    53 \div 2 & = 26 \cdots 1 & 6  \div 2 & = 3  \cdots 0 & 0.7 \times 2 & =0.4 + 1 & 0.6 \times 2 & =0.2 + 1 \\
    26 \div 2 & = 13 \cdots 0 & 3  \div 2 & = 1  \cdots 1 & 0.4 \times 2 & =0.8 + 0 & 0.2 \times 2 & =0.4 + 0 \\
    13 \div 2 & = 6  \cdots 1 & 1  \div 2 & = 0  \cdots 1 & 0.8 \times 2 & =0.6 + 1 & 0.4 \times 2 & =0.8 + 0
\end{align*}

\subsection{浮点数计算}
标准化IEEE浮点数$\pm 1.bbb\cdots b \times 2^p$,IEEE最近舍入法则:对53位,如果为0,则舍去,如果为1,则进位。
\begin{center}
    \begin{tabular}{cccc}
        \toprule
        精度     & 符号 & 指数 & 尾数 \\
        \hline
        单精度   & 1    & 8    & 23   \\
        双精度   & 1    & 11   & 52   \\
        长双精度 & 1    & 15   & 64   \\
        \bottomrule
    \end{tabular}
\end{center}
加减法时需要用0补齐到指数最大的数对应的位置。
\begin{align*}
    (8.3)_{10}=(1000.0\overline{1001})_2=1.0000\overline{1001}\times 2^3=1.\cdots 110011010                & \times 2^3 \\
    (7.3)_{10}=(111.0\overline{1001})_2=1.110\overline{1\textcolor{red}{001}}\times 2^2=1.\cdots 100110011 & \times 2^2 \\
    \text{(向52位之后的位置补齐)}(8.3)_{10}-(7.3)_{10}=1.00001001\cdots 10011010\textcolor{blue}{0}      & \times 2^3 \\
    \text{(对齐,保留52位之后的数字)}-0.11101001\cdots 10011001\textcolor{red}{1}                        & \times 2^3 \\
    \text{(52位之后的数字也要进行计算)}=0.00100000\cdots 00000000\textcolor{red}{1}                      & \times 2^3 \\
    \text{(移动到0位为1,舍入到52位)}=1.00000000\cdots 00000100\textcolor{white}{0}                      & \times 2^0
\end{align*}

\section{解方程}

\subsection{二分法}
初始区间$[a,b]$使$f(a)f(b)<0$,取中点$c=(a+b)/2$,计算$f(c)$,若$f(a)f(c)<0$,下一迭代区间为$[a,c]$,若$f(c)f(b)<0$,下一迭代区间为$[c,b]$,输出近似解为$x_a=(a+b)/2$。

\subsection{不动点迭代}
将方程$f(x)=0$变换为$g(x)=x$,可以达到线性收敛。
\begin{align*}
    x_{i+1} & =g(x_i) & \lim_{i\rightarrow \infty}\frac{e_{i+1}}{e_i}=S=|g'(r)| & <1
\end{align*}
牛顿法为特殊的不动点迭代方法,可以达到二次收敛。
\begin{align*}
    x_{i+1} & =x_i-\frac{f(x_i)}{f(x_{i+1})} & \lim_{i\rightarrow \infty}\frac{e_{i+1}}{e_i^2} & =\left| \frac{f''(r)}{2f'(r)} \right| & \lim_{i\rightarrow \infty}\frac{e_{i+1}}{e_i} & =\frac{m-1}{m}
\end{align*}

\subsection{误差}
近似解$x_a$的前向误差为$|r-x_a|$,后向误差为$|f(x_a)|$。如果$0=f(r)=f'(r)=\cdots=f^{(m-1)}(r)$,但是$f^{(m)}(r)\neq 0$,则$r$为$m$重根。

\section{解方程组}

\subsection{高斯消元法}
高斯消元的操作次数:$\frac{2}{3}n^3$。

\subsection{LU分解}
LU分解的操作次数:$\frac{2}{3}n^3$,每次回代的操作次数:$n^2$。

\subsection{PA=LU分解}
部分主元:每一行根据第一列元素从大到小排序,避免湮灭问题。
\begin{align*}
    \begin{bmatrix}
        2 & 3 \\
        3 & 2
    \end{bmatrix}
    \begin{bmatrix}
        x_1 \\
        x_2
    \end{bmatrix}
     & =
    \begin{bmatrix}
        4 \\
        1
    \end{bmatrix}
     & P & =
    \begin{bmatrix}
        0 & 1 \\
        1 & 0
    \end{bmatrix}
     & L & =
    \begin{bmatrix}
        1           & 0 \\
        \frac{2}{3} & 1
    \end{bmatrix}
\end{align*}
\begin{align*}
    A=
    \begin{bmatrix}
        2 & 3 \\
        3 & 2
    \end{bmatrix}
    \rightarrow ① \leftrightarrow ② \rightarrow
    \begin{bmatrix}
        3 & 2 \\
        2 & 3
    \end{bmatrix}
    \rightarrow ②-①\times\frac{2}{3}\rightarrow
    \begin{bmatrix}
        3 & 2           \\
        0 & \frac{5}{3}
    \end{bmatrix}
    =U
\end{align*}
\begin{align*}
    Lc=Pb\Rightarrow
    \begin{bmatrix}
        1           & 0 \\
        \frac{2}{3} & 1
    \end{bmatrix}
    \begin{bmatrix}
        c_1 \\
        c_2
    \end{bmatrix}
     & =
    \begin{bmatrix}
        0 & 1 \\
        1 & 0
    \end{bmatrix}
    \begin{bmatrix}
        4 \\
        1
    \end{bmatrix}
     &
    Ux=c \Rightarrow
    \begin{bmatrix}
        3 & 2           \\
        0 & \frac{5}{3}
    \end{bmatrix}
    \begin{bmatrix}
        x_1 \\
        x_2
    \end{bmatrix}
     & =
    \begin{bmatrix}
        c_1 \\
        c_2
    \end{bmatrix}
\end{align*}

\subsection{迭代方法}
要求矩阵$A$为严格对焦占优矩阵,$|a_{ii}|>\sum_{i\neq j}|a_{ij}|$,$D$表示$A$的主对角线矩阵,$L$表示$A$的下三角矩阵,$U$表示$A$的上三角矩阵。

雅可比方法
\begin{align*}
    x_{k+1}=D^{-1}(b-(L+U)x_k)
\end{align*}

高斯-赛德尔方法
\begin{align*}
    x_{k+1}=D^{-1}(b-Ux_k-Lx_{k+1})
\end{align*}

连续过松弛$\omega>1$,$\omega=1$时,退化为高斯-赛德尔方法
\begin{align*}
    x_{k+1}=(\omega L+D)^{-1}[(1-\omega)\mathrm{d}x_k-\omega Ux_k-Lx_{k+1}]+\omega(D+\omega L)^{-1}b
\end{align*}

\subsection{误差}
近似解$x_a$的余项为$r=b-Ax_a$,前向误差为$\|x-x_a\|$,后向误差为$\|b-Ax_a\|_\infty$。

\section{插值}

\subsection{拉格朗日插值}
对$n$个数据点$(x_1,y_1),\cdots,(x_n,y_n)$,有$n-1$阶多项式
\begin{align*}
    L_k(x)     & =\frac{(x-x_1)\cdots(x-x_{k-1})(x-x_{k+1})\cdots(x-x_n)}{(x_k-x_1)\cdots(x_k-x_{k-1})(x_k-x_{k+1})\cdots(x_k-x_n)}                      \\
    P_{n-1}(x) & =y_1L_1(x)+\cdots+y_nL_n(x)                                                                                                             \\
    P_2(x)     & =y_1\frac{(x-x_2)(x-x_3)}{(x_1-x_2)(x_1-x_3)}+y_2\frac{(x-x_3)(x-x_1)}{(x_2-x_3)(x_2-x_1)}+y_3\frac{(x-x_1)(x-x_2)}{(x_3-x_1)(x_3-x_2)}
\end{align*}

\subsection{牛顿商差}
\noindent
\begin{minipage}{0.42\linewidth}
    \begin{center}
        \begin{tabular}{c|ccc}
            $x_1$ & $f[x_1]$ &                 &                        \\
                  &          & $f[x_1 \, x_2]$ &                        \\
            $x_2$ & $f[x_2]$ &                 & $f[x_1 \, x_2 \, x_3]$ \\
                  &          & $f[x_2 \, x_3]$ &                        \\
            $x_3$ & $f[x_3]$ &                 &
        \end{tabular}
    \end{center}
\end{minipage}
\begin{minipage}{0.5\linewidth}
    \begin{align*}
        f[x_k]                       & =f(x_k)                                                      \\
        f[x_k \, x_{k+1}]            & =\frac{f[x_{k+1}]-f[x_k]}{x_{k+1}-x_k}                       \\
        f[x_k \, x_{k+1} \, x_{k+2}] & =\frac{f[x_{k+1} \, x_{k+2}]-f[x_k \, x_{k+1}]}{x_{k+2}-x_k} \\
        P_{n-1}(x)=\sum_{i=1}^n      & f[x_1\cdots x_i](x-x_1)\cdots(x-x_{i-1})
    \end{align*}
\end{minipage}


\subsection{切比雪夫插值}
在区间$[a,b]$内,切比雪夫插值节点
\begin{align*}
    x_i & =\frac{a+b}{2}+\frac{b-a}{2}\cos{\frac{(2i-1)\pi}{2n}} & i=1,2,\cdots,n
\end{align*}
使以下不等式在$[a,b]$内成立
\begin{align*}
    \left| (x-x_i)\cdots(x-x_n) \right|\leqslant \frac{\left( \frac{b-a}{2} \right)^n}{2^{n-1}}
\end{align*}

\subsection{插值误差}
插值误差公式,以及误差上界,其中$x_{min}<c<x_{max}$,
\begin{align*}
    f(x)-P(x)                & =\frac{(x-x_1)\cdots(x-x_n)}{n!}f^{(n)}(c)                                             \\
    \left| f(x)-P(x) \right| & \leqslant\frac{\left| (x-x_1)\cdots(x-x_n) \right|}{n!}\left| f^{(n)}(x) \right|_{max}
\end{align*}

\section{最小二乘法}

\subsection{不一致方程组}
对于不一致系统$Ax=b$,求解$A^TA\bar{x}=A^Tb$

\subsection{数据拟合}
多项式(线性)模型,周期型数据,指数模型的线性化。
\begin{align*}
    y & =c_1+c_2t+c_3t^2+\cdots                                & y & =c_1+c_2\cos{2\pi t}+c_3\sin{2\pi t}                           \\
    y & =c_1\mathrm{e}^{c_2t} \Rightarrow \ln{y}=\ln{c_1}+c_2t & y & =c_1t\mathrm{e}^{c_2t} \Rightarrow \ln{y}-\ln{t}=\ln{c_1}+c_2t
\end{align*}

\subsection{误差}
最小二乘解$\bar{x}$的余项为$r=b-A\bar{x}$,后向误差(二范数误差,欧式长度)为
\begin{align*}
    \left\| r \right\|_2=\sqrt{r_1^2+\cdots+r_m^2}
\end{align*}
平方误差,均方根误差,以及三者直接的关系分别为
\begin{align*}
    SE & =r_1^2+\cdots+r_m^2 & RMSE & =\sqrt{\frac{r_1^2+\cdots+r_m^2}{m}} & RMSE & =\frac{\sqrt{SE}}{\sqrt{m}}=\frac{\left\| r \right\|_2}{\sqrt{m}}
\end{align*}

\subsection{QR分解}
使用格拉姆-斯密特正交找出$A=(A_1|\cdots|A_n)$的消减QR分解
\begin{align*}
    y_j & =A_j-q_1(q_1^TA_j)-q_2(q_2^TA_j)-\cdots--q_{j-1}(q_{j-1}^TA_j) & q_j=\frac{y_j}{\left\| y_j \right\|_2}
\end{align*}
\begin{align*}
    \begin{cases}
        r_{jj}=\left\| y_j \right\|_2 \\
        r_{ij}=q_i^TA_j
    \end{cases}
    A=(q_1|\cdots|q_n)
    \begin{bmatrix}
        r_{11} & \cdots & r_{1n} \\
               & \ddots & \vdots \\
               &        & r_{nn}
    \end{bmatrix}
    =QR
\end{align*}
对$A=(A_1|\cdots|A_n)$补齐$(m-n)$个向量$[1,0,\cdots,0]^T$,计算$q_{n+1},\cdots,q_{n+m}$,得到完全QR分解,并通过回代$R\bar{x}=Q^Tb$,求解最小二乘问题(虚线以下部分直接扔掉)
\begin{align*}
    A=(q_1|\cdots|q_m)
    \begin{bmatrix}
        r_{11} & \cdots & r_{1n} \\
               & \ddots & \vdots \\
               &        & r_{nn} \\
        \cdashline{1-3}[0.8pt/2pt]
        0      & \cdots & 0      \\
        \vdots &        & \vdots \\
        0      & \cdots & 0
    \end{bmatrix}
     & =QR
     &
    \begin{bmatrix}
        r_{11} & \cdots & r_{1n} \\
               & \ddots & \vdots \\
               &        & r_{nn} \\
        \cdashline{1-3}[0.8pt/2pt]
        0      & \cdots & 0      \\
        \vdots &        & \vdots \\
        0      & \cdots & 0
    \end{bmatrix}
    \begin{bmatrix}
        x_1    \\
        \vdots \\
        x_n
    \end{bmatrix}
     & =
    \begin{bmatrix}
        q_1^T  \\
        \vdots \\
        q_m^T
    \end{bmatrix}
    \begin{bmatrix}
        b_1    \\
        \vdots \\
        b_n
    \end{bmatrix}
    =
    \begin{bmatrix}
        d_1     \\
        \vdots  \\
        d_n     \\
        \cdashline{1-1}[0.8pt/2pt]
        d_{n+1} \\
        \vdots  \\
        d_{n+m}
    \end{bmatrix}
\end{align*}

\section{数值微分积分}

\subsection{数值微分}
二点前向差分公式(一阶导一阶公式),三点中心差分公式(一阶导二阶公式)
\begin{align*}
    f'(x) & =\frac{f(x+h)-f(x)}{h}-\frac{h}{2}f''(c) & f'(x)=\frac{f(x+h)-f(x-h)}{2h}-\frac{h^2}{6}f'''(c)
\end{align*}
二阶导数的三点中心差分公式(二阶导二阶公式),都可以由泰勒展开推导而来。
\begin{align*}
    f''(x) & =\frac{f(x-h)-2f(x)+f(x+h)}{h^2}-\frac{h^2}{12}f^{(4)}(c)                                     \\
    f(x+h) & =f(x)+hf'(x)+\frac{h^2}{2}f''(x)+\frac{h^3}{3!}f'''(x)+\cdots+\frac{h^k}{k!}f^{(k)}(x)+\cdots
\end{align*}

\subsection{外推}
对$n$阶近似公式$F_n(x)$外推,至少可以得到$(n+1)$阶近似$Q$的公式
\begin{align*}
    Q \approx \frac{2^nF(h/2)-F(h)}{2^n-1}
\end{align*}

\subsection{数值积分}
梯形法则与复合梯形法则
\begin{align*}
    \int_{x_0}^{x_1} f(x) \,\mathrm{d}x & = \frac{h}{2}(y_0+y_1)-\frac{h^3}{12}f''(c)                           & h & =x_1-x_0       \\
    \int_{a}^{b} f(x) \,\mathrm{d}x     & = \frac{h}{2}(y_0+y_m+2\sum_{i=1}^{m-1}y_i)-\frac{(b-a)h^2}{12}f''(c) & h & =\frac{b-a}{m}
\end{align*}
辛普森法则与复合辛普森法则
\begin{align*}
    \int_{x_0}^{x_3} f(x) \,\mathrm{d}x & = \frac{h}{3}(y_0+4y_1+y_2)-\frac{h^5}{90}f^{(4)}(c) \qquad \qquad h=x_1-x_0=x_2-x_1                                            \\
    \int_{a}^{b} f(x) \,\mathrm{d}x     & = \frac{h}{3}(y_0+y_{m2}+4\sum_{i=1}^{m}y_{2i-1}+2\sum_{i=1}^{m-1}y_{2i})-\frac{(b-a)h^4}{180}f^{(4)}(c) \quad h=\frac{b-a}{2m}
\end{align*}
中点法则与复合中点法则
\begin{align*}
    \int_{x_0}^{x_1} f(x) \,\mathrm{d}x & =hf(x0+\frac{h}{2})+\frac{h^3}{24}f''(c)                       & h & =x_1-x_0       \\
    \int_{a}^{b} f(x) \,\mathrm{d}x     & =h\sum_{i=0}^{m-1}f(x_i+\frac{h}{2})+\frac{(b-a)h^2}{24}f''(c) & h & =\frac{b-a}{m}
\end{align*}

\subsection{龙贝格积分}
添加数据进行扩展,直到达到指定的精度,龙贝格积分是对复合梯形法则的外推。
\begin{align*}
    h_j & =\frac{b-a}{2^{j-1}} & R_{11} & =\frac{h_1}{2}(f(a)+f(b)) & R_{j1} & =\frac{1}{2}R_{j-1,1}+h_j\sum_{i=1}^{2^{j-2}}f(a+(2i-1)h_j)
\end{align*}
\begin{align*}
           & \begin{matrix}
        R_{11} &        &        &        \\
        R_{21} & R_{22} &        &        \\
        R_{31} & R_{32} & R_{33} & \ddots
    \end{matrix}                      &
    R_{jk} & =\frac{4^{k-1}R_{j,k-1}-R_{j-1,k-1}}{4^{k-1}-1} & R_{jj}\text{即为定积分的}2j\text{阶近似}
\end{align*}

\subsection{高斯积分}
\begin{center}
    \begin{tabular}{c|c|c}
        \toprule
        $n$阶 & 根$x_i$                                                      & 系数$c_i$ \\
        \hline
        $2$   & \makecell{$-\sqrt{\frac{1}{3}}=-0.577350269189$                          \\$\sqrt{\frac{1}{3}}=0.577350269189$}       &  \makecell{$1=1.000000000000$\\$1=1.000000000000$}         \\
        \hline
        $3$   & \makecell{$-\sqrt{\frac{3}{5}}=-0.774596669241$                          \\$0=0.000000000000$\\$\sqrt{\frac{3}{5}}=0.774596669241$}&\makecell{$\frac{5}{9}=0.555555555556$\\$\frac{8}{9}=0.888888888889$\\$\frac{5}{9}=0.555555555556$}\\
        \hline
        $4$   & \makecell{$-\sqrt{\frac{15+2\sqrt{30}}{35}}=-0.861136311594$             \\$-\sqrt{\frac{15-2\sqrt{30}}{35}}=-0.339981043585$\\$\sqrt{\frac{15-2\sqrt{30}}{35}}=0.339981043585$\\$\sqrt{\frac{15+2\sqrt{30}}{35}}=0.861136311594$}&\makecell{$\frac{90-5\sqrt{30}}{180}=0.347854845137$\\$\frac{90+5\sqrt{30}}{180}=0.652145154863$\\$\frac{90+5\sqrt{30}}{180}=0.652145154863$\\$\frac{90-5\sqrt{30}}{180}=0.347854845137$}\\
        \bottomrule
    \end{tabular}
\end{center}
对于积分区间不在$[-1,1]$上的积分,利用换元归一化到$[-1,1]$。s
\begin{align*}
    \int_{-1}^{1} f(x) \,\mathrm{d}x \approx & =\sum_{i=1}^{n}c_if(x_i) & \int_{a}^{b} f(x) \,\mathrm{d}x & = \int_{-1}^{1} f\left( \frac{(b-a)t+b+a}{2} \right)\frac{b-a}{2} \,\mathrm{d}t
\end{align*}

\section{常微分方程}

\subsection{初值问题}
欧拉方法,中点方法,显示梯形方法,$k$阶泰勒方法,偏导展开
\begin{align*}
    w_{i+1} & =w_i+hf(t_i,w_i),\,\,\,\,\, w_{i+1}=w_i+hf\left(t_i+\frac{h}{2},w_i+\frac{h}{2}f(t_i,w_i)\right)                                                       \\
    w_{i+1} & =w_{i}+\frac{h}{2}f(t_i,w_i)+\frac{h}{2}f\left(t_i+h,w_i+hf(t_i,w_i)\right)                                                                 \\
    w_{i+1} & =w_i+hf(t_i,w_i)+\frac{h^2}{2}f'(t_i,w_i)+\cdots+\frac{h^k}{k!}f^{(k-1)}(t_i,w_i)                                                           \\
    w_{i+1} & =w_i+hf(t_i,w_i)+\frac{h^2}{2}\left[ \frac{\partial f}{\partial t}(t_i,w_i)+\frac{\partial f}{\partial w}(t_i,w_i)f(t_i,w_i) \right]+\cdots
\end{align*}
二阶龙格库塔,第一种解为显示梯形方法,第二种解为中点方法
\begin{align*}
    w_{i+1} & =w_i+af(t_i,w_i)+bf\left(t_i+\alpha h,w_i+\beta f(t_i,w_i)\right)                                                           \\
    w_{i+1} & =w_i+(a+b)f(t_i,w_i)+b\alpha h\frac{\partial f}{\partial t}(t_i,w_i)+b\beta\frac{\partial f}{\partial w}(t_i,w_i)f(t_i,w_i) \\
            &
    \begin{cases}
        a+b=h \\
        b\alpha h=b\beta=\frac{h^2}{2}
    \end{cases}
    \Rightarrow
    \begin{cases}
        a=b=\frac{h}{2} \\
        \alpha=1,\beta=h
    \end{cases}
    \text{or\quad}
    \begin{cases}
        a=0,b=h \\
        \alpha=\frac{1}{2},\beta=\frac{h}{2}
    \end{cases}
\end{align*}
四阶龙格库塔\qquad$w_{i+1}=w_i+\frac{h}{6}(s_1+2s_2+2s_3+s_4)$\qquad$s_1=f(t_i,w_i)$
\begin{align*}
    s_2 & =f\left( t_i+\frac{h}{2},w_i+\frac{h}{2}s_1 \right) & s_3 & =f\left( t_i+\frac{h}{2},w_i+\frac{h}{2}s_2 \right) & s_4 & =f\left( t_i+h,w_i+hs_3 \right)
\end{align*}

\subsection{边值问题}
利用差分公式替换微分方程中的导数,组装矩阵,估计初值,解方程组。
\begin{align*}
    \begin{cases}
        y''=4y+t^2 \\
        y(a)=y_a   \\
        y(b)=y_b
    \end{cases}
    \Rightarrow
    \begin{cases}
        y_a+(-4h^2-2)w_1+w_2+h^2t_1^2=0         & t_1=a+h\text{处}          \\
        w_{i-1}+(-4h^2-2)w_i+w_{i+1}+h^2t_i^2=0 & \text{非边界}t_i\text{处} \\
        w_{n-1}+(-4h^2-2)w_n+y_b+h^2t_n^2=0     & t_n=b-h\text{处}
    \end{cases} \\
    \begin{bmatrix}
        -4h^2-2 & 1       & 0 & \cdots & 0 & 0       & 0       \\
        1       & -4h^2-2 & 1 &        & 0 & 0       & 0       \\
        0       & 1       &   & \ddots & 0 & 0       & 0       \\
        \vdots  &         &   & \ddots &   &         & \vdots  \\
        0       & 0       & 0 & \ddots &   & 1       & 0       \\
        0       & 0       & 0 &        & 1 & -4h^2-2 & 1       \\
        0       & 0       & 0 & \cdots & 0 & 1       & -4h^2-2
    \end{bmatrix}
    \begin{bmatrix}
        w_1     \\
        w_2     \\
        w_3     \\
        \vdots  \\
        w_{n-2} \\
        w_{n-1} \\
        w_n
    \end{bmatrix}
    =
    \begin{bmatrix}
        -h^2t_1^2-y_a \\
        -h^2t_2^2     \\
        -h^2t_3^2     \\
        \vdots        \\
        -h^2t_{n-2}^2 \\
        -h^2t_{n-1}^2 \\
        -h^2t_n^2-y_b
    \end{bmatrix}
\end{align*}
有限差分法$n$个区间,区间大小$h=1/(n+1)$,并可以得到n个方程
\begin{align*}
    \frac{w_{i-1}-2w_i+w_{i+1}}{h^2}-4w_i & =0 & w_0 & =a & w_n & =b
\end{align*}

\end{document}
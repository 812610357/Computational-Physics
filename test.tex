%!BIB program=biber

\documentclass[twocolumn]{article}
\usepackage{geometry}
\usepackage{fancyhdr}
\usepackage{amsmath}
\usepackage{cases}
\usepackage{graphicx}
\usepackage{subfigure} 
\usepackage{array}
\usepackage{unicode-math}
\usepackage[UTF8]{ctex}
\usepackage[hidelinks]{hyperref}
\usepackage{gbt7714}
\usepackage{doi}
\usepackage{xcolor}
\usepackage{algorithm2e} %伪代码宏
%\renewcommand{\vec}[1]{\boldsymbol{#1}}

\setmathfont{Cambria Math}
\geometry{a4paper,left=0.5cm,right=0.5cm,top=0.5cm,bottom=0.5cm}

\begin{document}
\small

\section{基础知识}

\subsection{二进制十进制转换}
例如$(53.7)_{10}=(110101.1\overline{0110})$
\begin{align*}
    53 \div 2 & = 26 \cdots 1 & 0.7 \times 2 & =0.4 + 1 \\
    26 \div 2 & = 13 \cdots 0 & 0.4 \times 2 & =0.8 + 0 \\
    13 \div 2 & = 6  \cdots 1 & 0.8 \times 2 & =0.6 + 1 \\
    6  \div 2 & = 3  \cdots 0 & 0.6 \times 2 & =0.2 + 1 \\
    3  \div 2 & = 1  \cdots 1 & 0.2 \times 2 & =0.4 + 0 \\
    1  \div 2 & = 0  \cdots 1 & 0.4 \times 2 & =0.8 + 0
\end{align*}

\subsection{浮点数计算}
标准化IEEE浮点数$\pm 1.bbb\cdots b \times 2^p$,IEEE最近舍入法则:对53位,如果为0,则舍去,如果为1,则进位。
\begin{table}[h]
    \centering
    \begin{tabular}{cccc}
        \hline
        精度     & 符号 & 指数 & 尾数 \\
        \hline
        单精度   & 1    & 8    & 23   \\
        双精度   & 1    & 11   & 52   \\
        长双精度 & 1    & 15   & 64   \\
        \hline
    \end{tabular}
\end{table}

加减法时需要用0补齐到指数最大的数对应的位置。
\begin{align*}
    (8.3)_{10}=(1000.0\overline{1001})_2=1.0000\overline{1001}\times 2^3=1.\cdots 110011010                & \times 2^3 \\
    (7.3)_{10}=(111.0\overline{1001})_2=1.110\overline{1\textcolor{red}{001}}\times 2^2=1.\cdots 100110011 & \times 2^2 \\
    (8.3)_{10}-(7.3)_{10}=1.00001001\cdots 10011010\textcolor{blue}{0}                                     & \times 2^3 \\
    -0.11101001\cdots 10011001\textcolor{red}{1}                                                           & \times 2^3 \\
    =0.00100000\cdots 00000000\textcolor{red}{1}                                                           & \times 2^3 \\
    =1.00000000\cdots 00000100\textcolor{white}{0}                                                         & \times 2^0
\end{align*}

\section{解方程}

\subsection{二分法}
\begin{algorithm}
    \KwIn{初始区间$[a,b]$使$f(a)f(b)<0$}
    \While{$(b-a)/2>TOL$}{
        $c=(a+b)/2$\\
        \If{$f(c) = 0$}{
            \textbf{break}
        }
        \eIf{$f(a)f(c)<0$}{
            $b=c$
        }{
            $a=c$
        }
    }
    \KwOut{近似根$r=(a+b)/2$}
\end{algorithm}

\subsection{不动点迭代}
\begin{align*}
    x_{i+1} & =g(x_i) & |g'(r)| & <1
\end{align*}

\subsection{牛顿方法}
\begin{align*}
    x_{i+1} & =x_i-\frac{f(x_i)}{f(x_{i+1})} & \lim_{i\rightarrow \infty}\frac{e_{i+1}}{e_i^2} & =\left| \frac{f''(r)}{2f'(r)} \right| & \lim_{i\rightarrow \infty}\frac{e_{i+1}}{e_i} & =\frac{m-1}{m}
\end{align*}

\section{解方程组}

\subsection{高斯消元法}
高斯消元的操作次数:$\frac{2}{3}n^3$

\subsection{LU分解}
LU分解的操作次数:$\frac{2}{3}n^3$,每次回代的操作次数:$n^2$

\subsection{PA=LU分解}
部分主元:每一行根据第一列元素从大到小排序,避免湮灭问题。
\begin{align*}
    \begin{bmatrix}
        2 & 3 \\
        3 & 2
    \end{bmatrix}
    \begin{bmatrix}
        x_1 \\
        x_2
    \end{bmatrix}
     & =
    \begin{bmatrix}
        4 \\
        1
    \end{bmatrix}
     & P & =
    \begin{bmatrix}
        0 & 1 \\
        1 & 0
    \end{bmatrix}
     & L & =
    \begin{bmatrix}
        1           & 0 \\
        \frac{2}{3} & 1
    \end{bmatrix}
\end{align*}
\begin{align*}
    A=
    \begin{bmatrix}
        2 & 3 \\
        3 & 2
    \end{bmatrix}
    \rightarrow ① \leftrightarrow ② \rightarrow
    \begin{bmatrix}
        3 & 2 \\
        2 & 3
    \end{bmatrix}
    \rightarrow ②-①\times\frac{2}{3}\rightarrow
    \begin{bmatrix}
        3 & 2           \\
        0 & \frac{5}{3}
    \end{bmatrix}
    =U
\end{align*}
\begin{align*}
    Lc=Pb\Rightarrow
    \begin{bmatrix}
        1           & 0 \\
        \frac{2}{3} & 1
    \end{bmatrix}
    \begin{bmatrix}
        c_1 \\
        c_2
    \end{bmatrix}
    &=
    \begin{bmatrix}
        0 & 1 \\
        1 & 0
    \end{bmatrix}
    \begin{bmatrix}
        4 \\
        1
    \end{bmatrix}
    &
    Ux=c \Rightarrow
    \begin{bmatrix}
        3 & 2           \\
        0 & \frac{5}{3}
    \end{bmatrix}
    \begin{bmatrix}
        x_1 \\
        x_2
    \end{bmatrix}
    &=
    \begin{bmatrix}
        c_1 \\
        c_2
    \end{bmatrix}
\end{align*}

\subsection{迭代方法}
要求矩阵$A$为严格对焦占优矩阵,$|a_{ii}|>\sum_{i\neq j}|a_{ij}|$
\end{document}